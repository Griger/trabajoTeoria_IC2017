\documentclass[10pt,a4paper]{article}

\usepackage[utf8]{inputenc}
\usepackage[spanish]{babel}
\usepackage{amsmath}
\usepackage{amsfonts}
\usepackage{amssymb}

\usepackage{color} %Use custom colors and give color to text
\usepackage{graphicx} %Include images
\usepackage{enumerate}

\author{\textbf{Gustavo Rivas Gervilla}}
\title{\textcolor{deepblue}{\textbf{Lógica difusa para la descripción de datos}}}
\date{}

%Custom colors
\definecolor{deepblue}{rgb}{0,0,0.5}

%Custom itemize bullet
\def\labelitemi{$\blacktriangleright$}

\begin{document}
\pagenumbering{gobble} %turn off page enumeration
\maketitle
\begin{center}
\includegraphics[scale=0.5]{img/decsai}
\end{center}

\newpage

\tableofcontents

\newpage
\pagenumbering{arabic} %turn on page enumeration
\section{Introducción}

Actualmente existe una gran cantidad de información que nos llega en distintos formatos. Por ejemplo nos pueden llegar en forma de tabla o forma de serie temporal. En caso de ser usuarios expertos podremos trabajar con estos datos y obtener las conclusiones que necesitemos para completar la tarea que estemos abordando, por ejemplo aplicando técnicas de minería de datos en caso de que poseyamos el conocimiento necesario para ello.\\

Pero la información no siempre va destinada a usuarios expertos con lo realizar una análisis de los datos en bruto para obtener una información que les sea de utilidad puede resultar muy complejo o imposible para ellos. Así que elebarorar una descripción linguística en forma de texto compresible por el usuario de los datos se ha convertido en los últimos tiempos en algo fundamental.\\

Hoy en día, la tarea de generar información intendible por los usuarios usando lenguaje natural (que es al fin y al cabo el que maneja el usuario en su día a día) ha sido abordada desde dos campos: el de la generación de languaje natural (\textbf{NLG}, por sus siglas en inglés) y el de la descripción lingüística de datos (\textbf{LDD}, por sus siglas en inglés). Pese a que estos dos campos son en inicio independiente la tendencia que están siguiendo el avance en ambos los llevará finalmente a converger.\\

El campo de la NLG se centra en la creación de textos que proporcionan información contenida en diversos formatos con el propósito de que esos textos sean indistinguibles, tanto como sea posible, de uno creado por humanos. Por el otro lado, el campo de la LDD, que aparece como un a de las principales aplicaciones de la \textbf{teoría de conjuntos difusos}, proporciona resúmenes o descripciones de conjuntos de datos empleando conceptos lingüísticos definidos en forma de conjuntos difusos y particiones difusas, lo que le permite tratar con la imprecisión inherente al lenguaje natural. Aquí se remarca una capacidad muy importante de los conjuntos difusos y es que nos permiten representar conceptos que no tienen una definición precisa y que están sujetos a la subjetividad de cada persona.\\

El campo de la NLG es más antiguo que el de la LDD, este campo comenzó a desarrollarse en los 80, y pese a tener ya un tiempo de desarrollo continúa siendo un campo de investigación abierto en muchos aspectos y no hay una técnica única para abordar los problemas de generación de lenguaje natural.\\

Por otro lado la descripción lingüística de datos trata de obtener descripciones concisas e informativas de \textit{datasets} numéricos y cubre un grupo de técnicas basadas en soft computing, como las variables lingüísticas o los cuantificadores y operadores difusos. Este campo comenzó a desarrollarse con fuerza a mediados de los 90 con el avance en el campo de los conjuntos difusos, dando lugar a nuevas aplicaciones en el enfoque descriptivo del \textit{data mining}.\\

\subsection{NLG}

John Bateman describió la generación de lenguaje natural como la rama del procesamiento de lenguaje natural que trata el problema crear automáticamente con una máquina textos en lenguaje natural.\\

La demanda de textos en lenguaje natural que proporcionen todo tipo de información está aumentando actualmente. Algunos ejemplos que podemos destacar de la NLG son \textbf{genración de partes meteorológicos en diversos idiomas a partir de datos meteorológicos}, \textbf{generación de cartas de respuesta a clientes}, \textbf{generación de informes sobre el estado de recien nacidos}. Pese a que nosotros queremos hablar sobre aplicaciones de la lógica difusa en descripción lingüística de datos, comentamos estas aplicaciones ya que como te comentó anteriormente y teniendo en cuenta que en este tipo de aplicaciones se trabaja con conceptos procedentes del lenguaje natural; más tarde o más temprano la lógica difusa enriquecerá y potenciará todas las aplicaciones que hemos enumerado.\\

A continuación vamos a dar una pequeña descripción de cómo se diseña un sistema de NLG, ya que algunas de las fases que vamos a enumerar tendrán su correspondencia con el diseño de la aplicación final que vamos a comentar.

\subsubsection{Diseño de un sistema de NLG}

El diseño de un sistema de generación de lenguaje natural es algo abierto en la que no hay una guía concreta que seguir. Dependerá del propio diseñador y del problema a resolver. No obstante podemos desglosar la tarea principal que afrontará cualquiera de estos sistemas (convertir unos datos de entrada en un texto) en una serie de tareas que podríamos considerar, en mayor o menor medida, como comunes a todos los sitemas de NLG:

\begin{itemize}
\item \textbf{Determinación del contenido:} decicir qué información se ha de comunicar por medio del texto a crear.
\item \textbf{Planificación del discurso:} dar un orden y una estructura al conjunto de mensajes a verbalizar.
\item \textbf{Agrupación de oraciones:} agrupar varios mensajes en una sola oración. Esta tarea no es siempre necesaria, en tanto en cuanto cada mensaje puede ser expresado por medio de una oración de forma individual.
\item \textbf{Lexicalización:} decidir que palabras y expresiones han de usarse para expresar los conceptos y relaciones del dominio que aparecen en los mensajes.
\item \textbf{Geración de las expresiones de referencia (referring expression):} seleccionar las palabras o expresiones que identifican las entidades del dominio. Aunque puede parecer una tarea similar a la anterior, en este caso buscamos discriminar cada entidad del dominio del resto; empleando para ello tanta información como sea necesaria (siempre que se disponga de ella claro).
\item \textbf{Realización lingúística:} aplicar reglas gramaticales para producir un texto que sea sintácticamente, morfológicamente y ortográficamente correcto a partir de los elementos anteriormente generados.
\end{itemize}

\section{LDD}

Este campo, que trata de construir descripciones de datasets empleando términos lingüísticos, surge con las ideas de \textbf{Lofti A. Zadeh} y \textbf{Ronald Yager}, lo cuáles presentaron la lógica difusa como una herramienta para realizar computación desde un punto de vista lingüístico. De estas ideas proviene el paradigma de la computación con palabras (CW, por sus siglas en inglés) y su evolución más moderna, la teoría computacional de conceptos (CTP, por sus siglas en inglés) que, según el propio Zadeh, \textit{añade a los sistemas de computación tradicionales dos importantes capacidades: (a) la capacidad de precisar el significado de las palabras y las proposiciones extraídas del lenguaje natural; y (b) la capacidad de razonar y computar con palabras y proposiciones precisadas}. Aunque han surgido muchas aproximaciones basadas en CW, la más prometedora es el resumen lingúístico de datos, el cual emplean sentencias cuantificadas de forma difusa para obtener resúmenes lingüísticos.\\

Algunos dominios de aplicación en los que se ha aplicado este paradigma es el flujo de entrada de pacientes a un centro de salud, consumo doméstico de electricidad, medición de la calidad de los andares de un individuo, actividad humana basada en acelerómetros del dispositivo móvil o en el terreno de la meteorología.\\

Este es un campo de reciente aparición con lo que encontrar una técnica general capaz de generar distintos tipos de descripciones lingüísticas para cualquier tiene de dominio de aplicación es una tarea aún por completar, aunque ya se han dado algunos pasos en esta dirección. Además también es importante la creación de criterios generales sobre cómo estructuras sentencias cuantificadas para obtener descripciones más complejas o cómo construir y evaluar descripciones lingüísticas. Con lo cual, al igual que sucede en el campo de la NLG, no hay un consenso sobre cómo ha de implementarse un sistema de LDD.

\subsection{Elementos en un enfoque de descripción lingüística de datos}


\section{Aplicación educativa a partir de expresiones de referencia}

\newpage

\section{Bibliografía}
\begin{enumerate}[{[}1{]}]
\item \textbf{On the role of linguistic descriptions of data in the building of natural language generation systems.} A. Ramos-Soto, A. Bugarín y S.Barro. Elsevier. Fuzzy sets and systems. \textit{13 de Julio de 2015}.
\item \textbf{On generating linguistic descriptions of time series.} Nicolás Marín y Daniel Sánchez. Elsevier. Fuzzy sets and systems. \textit{4 de Mayo de 2015}.
\end{enumerate}
\end{document}