\documentclass{beamer}
\usepackage[utf8]{inputenc}
\usepackage[spanish]{babel}
\usepackage{subfigure, wrapfig}
\usepackage{multirow}

\date{} %not show date

%beamer's options
\usetheme{Montpellier}
\author{Gustavo Rivas Gervilla}
\title{Lógica difusa en la descripción lingüística de datos}
\subtitle{LDD, series temporales y búsqueda de expresiones de referencia}
\institute{
	Escuela Técnica Superior de Ingeniería Informática y Telecomunicaciones	
}

\setbeamertemplate{section in toc shaded}[default][50]
%beamer's colors
\setbeamercolor{section in toc}{fg=orange}
\setbeamercolor{section in toc shaded}{fg=orange}
\setbeamercolor{structure}{fg=brown}
\setbeamercolor{title}{fg=brown}
\setbeamercolor{titlelike}{fg=brown}


\begin{document}
	\begin{frame}[plain]
		\titlepage
	\end{frame}
	
	\begin{frame}[plain]
		\frametitle{Contenido}
		\tableofcontents
	\end{frame}
	
	\AtBeginSection[]{
		\begin{frame}[plain]
			\frametitle{Contenido}
			\tableofcontents[currentsection]
		\end{frame}
	}
	
	\section{Introducción}
	
	\begin{frame}
		Gran cantidad de información $\longrightarrow$ usuarios no expertos.
		
		\begin{itemize}
			\item \textbf{NLG:} generación de textos que proporcionan información indistinguibles de los producidos por un humano.
			\item \textbf{LDD:} proporciona descripciones de conjuntos de datos empleando concepctos lingüísticos definidos mediante conjuntos difusos.
		\end{itemize}		
	\end{frame}
	
	\begin{frame}
		\begin{itemize}
			\item \textbf{Determinación del contenido:} decidir qué información se ha de comunicar por medio del texto a crear.
			\item \textbf{Planificación del discurso:} dar un orden y una estructura al conjunto de mensajes a verbalizar.
			\item \textbf{Agrupación de oraciones:} agrupar varios mensajes en una sola oración. (opcional)
			\item \textbf{Lexicalización:} decidir qué palabras y expresiones han de usarse.
			\item \textbf{Generación de las expresiones de referencia (referring expression):} seleccionar las palabras o expresiones que identifican las entidades del dominio.
			\item \textbf{Realización lingüística:} aplicar reglas gramaticales para producir un texto que sea sintácticamente, morfológicamente y ortográficamente correcto a partir de los elementos anteriormente generados.
\end{itemize}
	\end{frame}
	
	\section{LDD}
	
	\begin{frame}
		Surge a partir de las ideas de Zadeh y Yager. Usar la lógica difusa para realizar computación con palabras (CW).\\
				
		CW $\longrightarrow$ resumen lingüístico de datos.
		
		\begin{itemize}
			\item Flujo de pacientes.
			\item Consumo doméstico de electricidad.
			\item Actividad humana basada en acelerómetros de móviles.
			\item Meteorología.
		\end{itemize}
		
		Es un campo relativamente moderno con lo que no hay una técnica general para cualquier tipo de problema.
	\end{frame}
	
	\begin{frame}
		Dataset $\longrightarrow$ Extraer información  $\longrightarrow$ Representarla mediante conceptos lingüísticos.
		
		\begin{itemize}
			\item Los \textbf{datos de entrada}.
			\item \textbf{Variables lingüísticas}.
			\item \textbf{Cuantificadores difusos}.
			\item \textbf{Criterio de evaluación}.
		\end{itemize}
		
		Generar todas las posibilidades $\Rightarrow$ heurísticas y meta-heurísticas.
	\end{frame}
	
	\begin{frame}
		\begin{itemize}
			\item Campo muy teórico.
			\item Se pretende diseñar un framework de propósico general.
			\item Modelo granular de un fenómeno (GLMP) de Trivino y Sugeno: nodos con distintos niveles de abstracción interconectados entre sí.
		\end{itemize}		
	\end{frame}
	
	\section{Series temporales}
	
	\section{App}
	
	\subsection{Mecanismo de representación}
	\subsection{Problema}
	\subsection{Medida de calidad}
	\subsection{Algoritmo}
	
	
	
	
\end{document}